This is the getting started guide.\hypertarget{GettingStarted_HowToRun}{}\section{How to run}\label{GettingStarted_HowToRun}
To run the code using {\ttfamily  mpirun }, use the following command \-:


\begin{DoxyCode}
$ mpirun -n process\_number code\_path -f input\_file 
\end{DoxyCode}



\begin{DoxyItemize}
\item process\-\_\-number \-: The number of processes to run.
\item code\-\_\-path \-: The path to your executable code.
\item input\-\_\-file \-: The input file required for the code to run. \par

\begin{DoxyItemize}
\item (input file should be {\ttfamily .json} file)
\end{DoxyItemize}
\end{DoxyItemize}

{\bfseries Example} {\bfseries }  \$ mpirun -\/n 4 /usr/local/bin/mycode -\/f data/input.\-json 

{\bfseries Possble} command-\/line options {\bfseries } 


\begin{DoxyItemize}
\item -\/h \-: Help option
\item -\/f \-: \hyperlink{InputFile}{Input file} option (Required) (See \hyperlink{InputFile}{Input file})
\item -\/m \-: Method (See method)
\item -\/q \-: Quantity (See quantity)
\item -\/\-I \-: File path to set the position of bath spins (See bathfiles)
\item -\/s \-: File path to set the spin state of bath spins (See statefile)
\item -\/a \-: File path to set principal axies of the defects in the bath (See avaaxfile)
\item -\/\-S \-: File path to set the spin state of on-\/site spins of the defects in the bath (See exstatefile) 
\item -\/\-N \-: The number of state configuration to do time ensemble average (See nstate)
\item -\/\-B \-: Magnetic field in z direction (Unit \-: G) (See bfield)
\item -\/o \-: Result file path (See output)
\end{DoxyItemize}

{\bfseries About} options {\bfseries } 

The code read options twice \-:
\begin{DoxyItemize}
\item Read options from input file called as In-\/file options
\item Read optoins from command line called as Command-\/line options \par

\item Command-\/line options are \char`\"{}priority\char`\"{} \par

\end{DoxyItemize}\hypertarget{GettingStarted_Workflow}{}\section{Workflow}\label{GettingStarted_Workflow}
The C\-C\-E\-X software follow the below work process.

When you give the arguments in the command line, the code frist read the arguments. Note that if you don't give the input file accesible with -\/f option, then the code doesn't work.

Code read the all tags in input file and update the

 